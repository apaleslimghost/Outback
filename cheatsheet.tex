\documentclass[10pt, a4paper, twocolumn]{article}
\usepackage{outback}

\title{\uppercase{Outback}}
\subtitle{Cheat sheet}
\date{}
\pagenumbering{gobble}

\renewcommand{\maketitlehooka}{\vspace{-30pt}}
\renewcommand{\maketitlehookd}{\vspace{-40pt}}
\geometry{bottom=0.75cm, left=0.75cm, right=0.75cm, top=0.75cm}
\setlength{\parskip}{5pt}

\begin{document}
\twocolumn[
  \begin{@twocolumnfalse}
    \maketitle
  \end{@twocolumnfalse}
]

\section{Welcome to the Wasteland}

\subsection{Scrap}
The useful leftovers of dead vehicles and scavenged ruins is \emph{scrap}. In
the Wasteland, it's a currency, buying you gasoline, bullets, even vehicle parts
and training.

\section{Conflict resolution}
You and your opponent roll dice to see who succeeds. Your oppenent is usually
the GM, rolling for an NPC or the environment, or it could be another player.

Each of you rolls a \die{d12}, adding modifiers. If the result is 12 or higher,
write down ``12'', and roll again. Once your roll and modifier is less than 12,
stop rolling. Add up the twelves and the final roll, and that's the score. If
your score is greater than your opponent's, you succeeds; otherwise, your
opponent succeeds.

\section{Stats}
\begin{wraptable}[9]{r}{6ex}
  \small
\vspace*{-8ex}
\hspace*{-4.5ex}
\begin{tabular}{cc}
  Stat  & Mod \\
  \hline 
  0     & -5       \\
  1-2   & -4       \\
  3-4   & -3       \\
  5-6   & -2       \\
  7-8   & -1       \\
  9-10  &  0       \\
  11-12 & +1       \\
  13-14 & +2       \\
  15-16 & +3       \\
  17-18 & +4       \\
  19-20 & +5
\end{tabular}
\end{wraptable}

The three stats that sum up a character's ability are \stat{Brains} (mental
aptitude), \stat{Brawn} (physical strength) and \stat{Wit} (quick thinking).
Most of the time, these will be from 0 to 20, with a modifier from -5 to +5.

Your character might be \emph{trained} in certain skills. For rolls with these
skills, the modifier has an extra +1.

\subsection{Vehicle stats}
Vehicles have separate sets of stats: \stat{Speed}, \stat{Handling},
\stat{Acceleration}, \stat{Braking}, \stat{Ruggedness} and \stat{Weight}. A vehicle
is just the sum of its parts. Each vehicle part --- engine, transmission, tyres
etc.\ --- confers stats to the vehicle, and can be upgraded separately.

\stat{Speed} is different. It's not 0-20 like the others, but from 0 to your
vehicle's \stat{Maximum Speed}, in \emph{feet per second}. It changes as you
man\oe{}uvre your vehicle. Convert it to a 0-20 stat by dividing by 10, and to a
modifier from the stat in the usual way.

\subsection{Combined modifiers}
Sometimes, such as when rolling for Driving checks, modifiers from two sources
apply. Combine them by taking the average of the two modifiers. For example, if
your \stat{Wit} modifier is +4 and your vehicle's \stat{Handling} is +2, the
combined modifier is +3.

\section{Creating a character}
Generate your three \emph{stats} in one of the following ways. \textbf{Dice:}
roll \die{5d4} for each stat. \textbf{Points:} choose three from 6 to 17,
totalling no more than 40. \textbf{Array:} assign 9, 14 and 17 to your stats.
Next, choose three \emph{skills} from the list to become \emph{trained} in.
Rolls in trained skills add an extra +1 to the modifier. Finally, starting with
200 scrap, buy \emph{equipment} from the list (or choose one of the standard
loadouts).

\subsection{Hit points}
\emph{Hit points (HP)} are you and your vehicle's ability to keep going. A
character's \emph{Maximum HP} is equal to 3 times their \stat{Brawn} stat. A
vehicle's Maximum HP is 5 times its \stat{Ruggedness} stat.

\subsection{Advancement}
Once per day, while you are in the relative safety of a camp or stronghold, you
can \emph{advance} your character. Either increase one stat by 1 point,
costing scrap equal to 5 times your current stat total, or train one untrained
skill, costing 20 scrap for each skill already trained.

\section{Combat}
A combat roll is an offensive skill, such as \stat{Brawn (Melee weapons)} versus
a defensive skill, such as \stat{Wit (Agility)}. If the attacker succeeds, the
defender takes damage equal to the difference between the rolls, up to a maximum
specified by the weapon.

Time in combat is broken down into \emph{ticks}, an abstract time period that
can be taken to be roughly a second. In each tick, players declare actions,
which execute simultaneously.

\begin{wraptable}[5]{l}{8ex}
  \small
\vspace*{-4ex}
\hspace*{-4ex}
\begin{tabular}{cc}
  Length   & Ticks \\
  \hline 
  Instant  & 1     \\
  Quick    & 2     \\
  Steady   & 3     \\
  Slow     & 5     \\
  Long     & 8
\end{tabular}
\end{wraptable}

Every action has a \emph{cooldown} period, which is a number of ticks before you
can act again. This includes the current tick, so after an Instant action you
can move in the next tick.

\subsection{Vehiclular combat}
Vehicles are tracked on and off the road by three variables: \emph{position},
\emph{speed} and \emph{direction}. Position and direction are always relative.
During combat, the GM will keep track, and announce for each NPC combatant.

When making a driving-related die roll, combine your \stat{Wit (Driving)}
modifier with the stat modifier from the vehicle.

\subsubsection{Man\oe{}uvring}
\emph{Man\oe{}uvres} are actions that alter your speed, let you take corners,
and so on. They are taken like ordinary actions, except all man\oe{}uvres
execute simultaneously at the end of the tick. 

Most man\oe{}uvres do not require a roll to succeed. It's not difficult to push
down a gas pedal, but it does take skill to get the best out of your vehicle.
For some man\oe{}uvres, you roll dice to determine how great an effect they
have.

You do not always have to choose between action and man\oe{}uvre: even if you
have taken a non-man\oe{}uvre action, you can still take an Instant man\oe{}uvre
in the same tick. If you do not man\oe{}uvre, you must take the man\oe{}uvre
\emph{Coast}.

\subsubsection{Fuel}
Most man\oe{}uvres cost \emph{fuel}, listed in its description. You can only
perform a man\oe{}uvre if you have enough fuel, and once you are out of fuel,
you cannot man\oe{}uvre.

\subsubsection{Damage, afflictions and repair}
Attacks to a vehicle damage the vehicle, not its occupants, unless the attack
states otherwise.

Certain attacks can cause \emph{afflictions}. These are statuses such as
\emph{blowout} that adversely affect your vehicle's stats.

When a vehicle reaches zero hit points, it is afflicted by \emph{engine fire}
and cannot be man\oe{}uvred. If the fire is not dealt with in 5 ticks, the
engine explodes, killing any occupants.

TODO Repair

\subsection{Hand-to-hand \& ranged combat}
A hand-to-hand attack roll is the attacker's Brawn (Fisticuffs or Melee Weapons)
against the defender's Brawn (Armour) or Wit (Agility). A ranged weapon attack
roll is Brains (Ranged Weapons) versus Armour or Agility. 

With a melee weapon, or unarmed, you can only attack a combatant within 3 feet
of you, unless the weapon states otherwise. Ranged weapons specify their maximum
range. 

While on foot, as an Instant man\oe{}uvre, you can run a number of feet up to 10
plus your \stat{Brawn} modifier.
\end{document}