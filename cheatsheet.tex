\documentclass[10pt, a4paper, twocolumn]{article}
\usepackage{outback}

\title{\uppercase{Outback}}
\subtitle{Cheat sheet}
\date{}
\pagenumbering{gobble}

\renewcommand{\maketitlehooka}{\vspace{-30pt}}
\renewcommand{\maketitlehookd}{\vspace{-40pt}}
\geometry{bottom=0.75cm, left=0.75cm, right=0.75cm, top=0.75cm}
\setlength{\parskip}{5pt}

\begin{document}
\twocolumn[
  \begin{@twocolumnfalse}
    \maketitle
  \end{@twocolumnfalse}
]

\section{Conflict resolution}
You and your opponent roll dice to see who succeeds. Your oppenent could be the
GM, rolling for an NPC or the environment, or it could be another player.

Each of you rolls a \die{d12}, adding any appropriate modifiers. If the result
is 12 or higher, mark down ``12'' somewhere, and roll again. Once your roll and
modifier is less than 12, stop rolling. Add up all your twelves and your final
roll, and that's your score. If your score is greater than your opponent's, you
succeeds; otherwise, your opponent succeeds.

Depending on the roll, the difference between the scores could factor into the
result, so keep a note of it. 

\subsection{Round down}
When we divide numbers we \emph{round down}. Remove the fractional part from the
number, to round towards zero: 3{1/2} rounded down is 3, while -2{1/2} rounded
down is -2. 

\section{Stats}

\begin{wraptable}[9]{r}{6ex}
  \small
\vspace*{-8ex}
\hspace*{-4.5ex}
\begin{tabular}{cc}
  Stat  & Mod \\
  \hline 
  0     & -5       \\
  1-2   & -4       \\
  3-4   & -3       \\
  5-6   & -2       \\
  7-8   & -1       \\
  9-10  &  0       \\
  11-12 & +1       \\
  13-14 & +2       \\
  15-16 & +3       \\
  17-18 & +4       \\
  19-20 & +5
\end{tabular}
\end{wraptable}

The three stats that sum up a character's ability are \stat{Brains} (puzzles,
navigation, repair,  etc), \stat{Brawn} (physical strength, taking a hit,
running, etc) and \stat{Wit} (quick thinking, driving, bartering, etc).

These stats are ordinarily from 0 to 20; modifiers run from -5 to +5. Look up
modifiers in the table, or calculate by adding 1 to the stat and halving, then
subtracting 5.

Your character might be \emph{trained} in certain skills. For rolls with these
skills, add 1 to the stat modifier.

\subsection{Vehicle stats}
Vehicles have separate sets of stats: \stat{Speed}, \stat{Handling},
\stat{Acceleration}, \stat{Braking}, \stat{Ruggedness} and \stat{Weight}. A vehicle
is just the sum of its parts. Each part confers stats to the vehicle's total,
and can be upgraded separately.
.
\stat{Speed} is different. It's not 0-20 like the others, but from 0 to your
vehicle's \stat{Maximum Speed}. It changes as you man\oe{}uvre your vehicle.
Convert it to a 0-20 stat by dividing by 10 (just remove the last number), and
to a modifier from the stat in the usual way.

\subsection{Hit points}
\emph{Hit points (HP)} are you and your vehicle's ability to keep going. A
character's \emph{Maximum HP} is equal to 3 times their \stat{Brawn} stat. A
vehicle's Maximum HP is 5 times its \stat{Ruggedness} stat.

\subsection{Creating a character}
Generate your three stats in one of the following ways:\\
\textbf{Dice:} roll \die{5d4} for each stat\\
\textbf{Points:} choose three from 6 to 17, totalling no more than 40\\
\textbf{Array:} assign 9, 14 and 17 to your stats

Next, choose three skills from the list to become \emph{trained} in. Rolls in
trained skills add an extra +1 to the modifier.

Finally, TODO equipment

\subsection{Combined modifiers}
Sometimes, such as when rolling for Driving checks, you combine two
modifiers. To do this, take the average of the two modifiers (add them
together and half the total). For example, if your \stat{Wit}
modifier is +4 and your vehicle's \stat{Handling} is +2, the combined modifier
is +3.

\section{Combat}
Time in combat is broken down into \emph{ticks}, an abstract time period that
can be taken to be approximately one second. During a tick, all act
simultaneously. An \emph{action} could be making an attack, man\oe{}uvring your
vehicle, or something else. 

\begin{wraptable}[5]{l}{8ex}
  \small
\vspace*{-4ex}
\hspace*{-4ex}
\begin{tabular}{cc}
  Length   & Ticks \\
  \hline 
  Instant  & 1     \\
  Quick    & 2     \\
  Steady   & 3     \\
  Slow     & 5     \\
  Long     & 8
\end{tabular}
\end{wraptable}

Each action has a \emph{cooldown} period, which is a number of ticks before you
can take action again. This includes the current tick, so after an Instant
action you can move in the next tick. The possible cooldown lengths are shown in
the table. 

\subsection{Vehiclular combat}
Vehicles are tracked on and off the road by three variables: \emph{position},
\emph{speed} and \emph{direction}. Position and direction are always relative.
During combat, the GM will keep track, and announce for each NPC combatant.

When making a driving-related die roll, combine your Driving skill modifier with
the stat modifier from the vehicle.

\subsubsection{Man\oe{}uvring}
Altering your speed, taking corners, and so on are all done by special actions
called \emph{man\oe{}uvres}. They are taken like ordinary actions, but do not
have effect until the end of the tick. All man\oe{}uvres in a tick execute
simultaneously. 

Most man\oe{}uvres do not require a roll to succeed. It's not difficult to push
down a gas pedal, after all. It does take skill to squeeze out every bit of
performance from your vehicle. For some man\oe{}uvres, you roll dice to
determine how great an effect they have.

You do not always have to choose between action and man\oe{}uvre: if you have
already taken a non-man\oe{}uvre action, you can still take an Instant
man\oe{}uvre in the same tick.

If you do not man\oe{}uvre in a tick, you must take the man\oe{}uvre
\emph{Coast}.

\subsubsection{Fuel}
Most man\oe{}uvres cost \emph{fuel}. Each man\oe{}uvre lists how much fuel it
requires. If you do not have enough fuel to perform a man\oe{}uvre, it cannot be
taken. When you reach zero fuel, you must \emph{Coast}.

\subsubsection{Damage, afflictions and repair}
Attacks to a vehicle damage the vehicle, not its occupants, unless the attack
states otherwise.

Certain attacks can cause \emph{afflictions}. These are statuses such as
\emph{blowout} that adversely affect your vehicle's stats.

When a vehicle reaches zero hit points, it is afflicted by \emph{engine fire}
and cannot be man\oe{}uvred. If the fire is not dealt with in 5 ticks, the
engine explodes, killing any occupants.

TODO Repair

\subsection{Hand-to-hand \& ranged combat}
A hand-to-hand attack roll is the attacker's Brawn (Fisticuffs or Melee Weapons)
against the defender's Brawn (Armour) or Wit (Agility). If the attacker
succeeds, the defender takes damage equal to the difference between the rolls,
up to a maximum specified by the weapon. 

A ranged weapon attack roll is Brains (Ranged Weapons) versus Armour or Agility.

With a melee weapon, or unarmed, you can only attack a combatant within 3 feet
of you, unless the weapon states otherwise. Ranged weapons specify their maximum
range. 

While on foot, as an Instant man\oe{}uvre, you can run a number of feet up to 10
plus your \stat{Brawn} modifier.
\end{document}