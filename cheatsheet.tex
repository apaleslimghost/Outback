\documentclass[10pt, a4paper, twocolumn]{article}
\usepackage{outback}

\title{\uppercase{Outback}}
\subtitle{Cheat sheet}
\date{}
\pagenumbering{gobble}

\renewcommand{\maketitlehooka}{\vspace{-60pt}}
\renewcommand{\maketitlehookd}{\vspace{-30pt}\hrule\vspace{-10pt}}

\begin{document}
\maketitle

\section{Conflict resolution}
You and your opponent roll dice to see who succeeds. Your oppenent could be the
GM, rolling for an NPC or the environment, or it could be another player.

Each of you rolls a \die{d12}, adding any appropriate modifiers. If the result
is 12 or higher, mark down ``12'' somewhere, and roll again. Once your roll and
modifier is less than 12, stop rolling. Add up all your twelves and your final
roll, and that's your score. If your score is greater than your opponent's, you
succeeds; otherwise, your opponent succeeds.

Depending on the roll, the difference between the scores could factor into the
result, so keep a note of it. 

\subsection{Halving and rounding}
Sometimes, we need to half a number. We don't want fractions, so we
\emph{round down}. Positive numbers round \emph{towards zero}, i.e. 4.5
rounded down is 4. Negative numbers round \emph{away from zero}, i.e. -2.5
rounded down is -3.

\section{Stats}
The three stats that sum up a character's ability are \stat{Brains} (problem
solving, navigation, repair,  etc), \stat{Brawn} (physical strength, taking a
hit, running, jumping, etc) and \stat{Wit} (quick thinking, driving,
initiative, bartering, etc).

When making a roll that requires a particular skill, add your modifier for the
appropriate stat to each roll. The GM will decide which stat is appropriate,
depending on how you roleplay going about the task.

\begin{wraptable}[12]{r}{0.13\textwidth}
\vspace*{-3ex}
\hspace*{-3ex}
\begin{tabular}{cc}
  Stat  & Modifier \\
  \hline 
  0     & -5       \\
  1-2   & -4       \\
  3-4   & -3       \\
  5-6   & -2       \\
  7-8   & -1       \\
  9-10  &  0       \\
  11-12 & +1       \\
  13-14 & +2       \\
  15-16 & +3       \\
  17-18 & +4       \\
  19-20 & +5
\end{tabular}
\end{wraptable}

Vehicles have separate sets of stats: \stat{Handling}, \stat{Acceleration},
\stat{Speed}, \stat{Braking}, \stat{Armour} and \stat{Weight}. A vehicle is just
the sum of its parts. Each part confers stats to the vehicle's total, and can be
upgraded separately: swap out your engine, and your Acceleration could go up;
new tyres might improve your Handling.

These stats are ordinarily from 0 to 20; modifiers
run from -5 to +5. Calculate your modifier by subtracting 9 and halving
(rounding down), or by using the table.

\subsection{Combined modifiers}
Sometimes, such as when rolling for Driving checks, you combine two
modifiers. To do this, take the average of the two modifiers (i.e. add them
together and half the total), rounding down. For example, if your \stat{Wit}
modifier is +4 and your vehicle's \stat{Handling} is +2, the combined modifier is +3. 
+4 and -3 combined is 0.


\section{Dummy}
\lipsum{}
\end{document}