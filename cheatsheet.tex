\documentclass[10pt, a4paper, twocolumn]{article}
\usepackage{outback}
\usepackage{onepage}

\title{\uppercase{Outback}}
\subtitle{Cheat sheet}
\date{}

\begin{document}
\twocolumn[
  \begin{@twocolumnfalse}
    \maketitle
  \end{@twocolumnfalse}
]

\section{Welcome to the Wasteland}
The world died. All that's left is the Wasteland: a vast desert dotted with
strongholds of warring tribes. Only the mad survive.

\subsection{Scrap}
The useful leftovers of dead vehicles and scavenged ruins is \emph{scrap}. In
the Wasteland, it's a currency, buying you gasoline, bullets, even vehicle parts
and training.

\section{Conflict resolution}
You and your opponent roll dice to see who succeeds. Your oppenent is usually
the GM, rolling for an NPC or the environment, or it could be another player.

Each of you rolls a \die{d12}, adding modifiers. If the result is 12 or higher,
write down ``12'', and roll again. Once your roll and modifier is less than 12,
stop rolling. Add up the twelves and the final roll, and that's the score. If
your score is greater than your opponent's, you succeeds; otherwise, your
opponent succeed.

\section{Stats}
\begin{wraptable}[9]{r}{6ex}
  \small
\vspace*{-3.5ex}
\hspace*{-4.5ex}
\begin{tabular}{cc}
  Stat  & Mod \\
  \hline 
  0     & -5       \\
  1-2   & -4       \\
  3-4   & -3       \\
  5-6   & -2       \\
  7-8   & -1       \\
  9-10  &  0       \\
  11-12 & +1       \\
  13-14 & +2       \\
  15-16 & +3       \\
  17-18 & +4       \\
  19-20 & +5
\end{tabular}
\end{wraptable}

The three stats that sum up a character's ability are \stat{Brains},
\stat{Brawn} and \stat{Wit}. Most of the time, these will be from 0 to 20, with
a modifier from -5 to +5. 

Your character might be \emph{trained} in certain skills. For rolls with these
skills, the modifier has an extra +1.

\subsection{Vehicle stats}
Vehicles have separate sets of stats: \stat{Speed}, \stat{Handling},
\stat{Acceleration}, \stat{Brakes}, \stat{Ruggedness} and \stat{Weight}. Each
vehicle part confers stats to the vehicle, and can be upgraded separately. 

\stat{Speed} is different. It's not 0-20 like the others, but from 0 to your
vehicle's \stat{Maximum Speed}, in \emph{feet per second}. It changes as you
man\oe{}uvre your vehicle. Convert it to a 0-20 stat by dividing by 10, and to a
modifier from the stat in the usual way.

\subsection{Combined modifiers}
Sometimes, such as when rolling for driving checks, modifiers from two sources
apply. Combine them by taking the average of the two modifiers. For example, if
your \stat{Wit} modifier is +4 and your vehicle's \stat{Handling} is +2, the
combined modifier is +3.

\section{Creating a character}
Generate your stats in one of the following ways. \textbf{Dice:} roll \die{5d4}
for each stat. \textbf{Points:} choose three from 6 to 17, totalling no more
than 40. \textbf{Array:} assign 9, 14 and 17 to your stats. Next, choose three
skills from the list to become trained in. Finally, starting with 200 scrap, buy
equipment from the list (or choose one of the standard loadouts).

\subsection{Hit points}
A character's Maximum HP is equal to 3 times their \stat{Brawn} stat. A
vehicle's Maximum HP is 5 times its \stat{Ruggedness} stat.

\subsection{Advancement}
Once per day, while you are in the relative safety of a camp or stronghold, you
can \emph{advance} your character. Either increase one stat by 1 point,
costing scrap equal to 5 times your current stat total, or train one untrained
skill, costing 20 scrap for each skill already trained.

\section{Combat}
A combat roll is an offensive skill, such as \stat{Brawn (Melee weapons)} versus
a defensive skill, such as \stat{Wit (Agility)}. If the attacker succeeds, the
defender takes damage equal to the difference between the rolls, up to a maximum
specified by the weapon.

Time in combat is broken down into \emph{ticks}, an abstract time period that
can be taken to be roughly a second. In each tick, players declare actions,
which happen simultaneously.

\begin{wraptable}[5]{l}{8ex}
  \small
\vspace*{-4ex}
\hspace*{-4ex}
\begin{tabular}{cc}
  Length   & Ticks \\
  \hline 
  Instant  & 1     \\
  Quick    & 2     \\
  Steady   & 3     \\
  Slow     & 5     \\
  Long     & 8
\end{tabular}
\end{wraptable}

Every action has a \emph{cooldown}, which is a number of ticks before you can
act again. This includes the current tick, so after an \emph{Instant} action you
can move in the next tick. 

\subsection{Vehiclular combat}
Vehicles are tracked  by three variables: \emph{position}, \emph{speed} and
\emph{direction}. Position and direction are always relative. During combat, the
GM will keep track, and announce for each combatant.

When making a driving-related dice roll, combine your \stat{Wit (Driving)}
modifier with the stat modifier from the vehicle.

A vehicular attack is normally \stat{Wit (Driving) + Speed} versus \stat{Wit
  (Driving) + Ruggedness}.

\subsubsection{Man\oe{}uvring}
\emph{Man\oe{}uvres} are actions that alter your speed, let you take corners,
and so on. They are taken like ordinary actions, except all man\oe{}uvres
execute at the end of the tick.

Most man\oe{}uvres do not require a roll to succeed, but it does take skill to
get the best out of your vehicle. For some man\oe{}uvres, you roll dice to
determine how great an effect they have.

You do not always have to choose between action and man\oe{}uvre: even if you
have taken a non-man\oe{}uvre action, you can still take an Instant man\oe{}uvre
in the same tick. If you do not man\oe{}uvre, you must \emph{Coast}.

\subsubsection{Fuel}
Most man\oe{}uvres cost \emph{fuel}; how much depends on the man\oe{}uvre. You
can only man\oe{}uvre if you have enough fuel.

\subsubsection{Damage \& afflictions}
Attacks to a vehicle damage the vehicle, not its occupants, unless the attack
states otherwise. Certain attacks can cause \emph{afflictions}. These are
statuses such as \emph{blowout} that adversely affect your vehicle's stats.

When a vehicle reaches zero hit points, it is afflicted by \emph{engine fire}
and cannot be man\oe{}uvred. If the fire is not dealt with in 5 ticks, the
engine explodes, killing any occupants.

\subsection{Hand-to-hand \& ranged combat}
A hand-to-hand attack roll is the attacker's \stat{Brawn (Fisticuffs or Melee
  Weapons)} against the defender's \stat{Brawn (Armour)} or \stat{Wit
  (Agility)}. A ranged weapon attack roll is \stat{Brains (Ranged Weapons)}
versus Armour or Agility. 

With a melee weapon, or unarmed, you can only attack a combatant within 3 feet
of you, unless the weapon states otherwise. Ranged weapons specify their maximum
range. 

While on foot, as an Instant man\oe{}uvre, you can run a number of feet up to 10
plus your \stat{Brawn} modifier.
\end{document}