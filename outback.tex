\documentclass[10pt, a4paper, twocolumn]{book}
\usepackage{outback}

\title{\uppercase{Outback}}
\subtitle{An RPG system for the mad}
\author{Matt Brennan}
\date{}

\begin{document}
\frontmatter
\maketitle
\tableofcontents

\mainmatter
\chapter{Welcome to the wasteland}
After the end of civilization the only place left was the Wasteland. A trackless desert with little water, food, or resources, it is an inhospitable arid landscape. Very little grows or lives here and the human survivers are left fighting over the few resources for survival. Some shadows of civilization have been created, mostly roving bands of thugs, scavengers, or traders. 
\section{Playing the game}
\subsection{Rolling dice}
When you undertake a difficult action, you make a die roll in one of your stats.
Roll a \die{d20}, and add your modifier for the stat. If the result is
\emph{less} than 20, that's your score. If it's \emph{more} than 20, well, Lady
Luck just found a friend. Write ``20'' down somewhere, and roll the same again. Each
time your die plus your modifier is more than 20, jot down another 20 and keep going.
When it's finally less than 20, tot up all them twenties and add your final roll
and modifier. \emph{That's} your score, and a mighty fine one.

You'll be rolling against the GM's or another player's dice, or the static score
of an environment. Your opponent doesn't get to reroll their dice: one roll and
modifier and that's it. Subtract the opposing score. If it's less than or equal
to zero then it was all for nothing: your task fails. More than zero and you
succeed. Some rolls have effects that use the difference between the dice, so
keep a note of the number.

\subsection{Rolls for driving}
Vehicles have their own set of stats, which depend on how the vehicle is
outfitted, its condition \emph{et cetera}. When you perform a driving-related
task (for example manoeuvring, braking or ramming), your modifier for the roll
is your \stat{Driving} modifier plus the car's relevant modifier.

\subsection{Stats}
A stat is a number from 0 to 20 that describes your aptitude in a particular
skill or a facet of your vehicle's capability. The modifier for a stat is
given by the table below (or by subtracting 9 and dividing by 2, rounding
down).

\begin{wraptable}{l}{0.13\textwidth}
\vspace*{-10pt}
\begin{tabular}{cc}
  Stat  & Modifier \\
  \hline 
  0     & -5       \\
  1-2   & -4       \\
  3-4   & -3       \\
  5-6   & -2       \\
  7-8   & -1       \\
  9-10  &  0       \\
  11-12 & +1       \\
  13-14 & +2       \\
  15-16 & +3       \\
  17-18 & +4       \\
  19-20 & +5
\end{tabular}
\vspace*{-10pt}
\end{wraptable}

The stats for characters are \stat{Brawn}, \stat{Brains}, \stat{Armour}, \stat{Agility},
\stat{Driving} and \stat{Repair}. For vehicles, there's \stat{Speed},
\stat{Acceleration}, \stat{Braking}, \stat{Handling}, \stat{Ruggedness} and \stat{Weight}.

Depending on environmental conditions, injuries, wear \&
tear and other factors, there could be temporary modifiers that also apply to
rolls. For example, on rough terrain, \stat{Handling} rolls will have a penalty
of -2.

\chapter{Combat}
\section{Ticks, Actions \& Man\oe{}uvres}
Time in combat is broken down into \emph{ticks}. A tick represents one in-game
second. An \emph{action} is something you can do in a tick

\begin{wrapfigure}{l}{0.13\textwidth}
  \textbf{}\\*
  Hello
\end{wrapfigure}

rosnietarsoni oiresnat rsointa

\end{document}