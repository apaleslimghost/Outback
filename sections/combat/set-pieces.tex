% !TEX root = ../../outback-basic-and-extended.tex

Time in combat is broken down into \emph{setpieces}, which consist of 24 \emph{ticks}. At the start of a setpiece, each character rolls a single \die{d12}, adding their \stat{Wit} plus 6. That's their \emph{starting tick}.

To keep track of ticks, divide a sheet of paper into sections numbered \textbf{1} to \textbf{24}, and place tokens for each character at their starting tick. Play progresses from the top of the setpiece. The GM calls out each tick number, and the characters at that number can act, in descending order of \stat{Wit}.

On a character's turn, their player describes their a \emph{sequence} of actions. For example, \textbf{``\emph{I veer my vehicle over to the rig, open my door and climb on top, jump across to the rig, and shoot my Glock at the driver}''} describes a sequence of four actions.

Each action in the sequence takes you down the sheet by two ticks. Once you're below tick 1, you can no longer act. So, if you're on tick 5 or 6, you can take a sequence of 3 actions; if you're on tick 1 or 2 you can only take one action.

Some actions can only be taken as the \emph{last action in a setpiece}. This is the action that takes you below tick 1. Once every character is below tick 1, the setpiece ends. Characters roll again for starting tick and the next setpiece begins.
