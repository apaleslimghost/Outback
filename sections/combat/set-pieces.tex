% !TEX root = ../../outback-basic-and-extended.tex

Time in combat is broken down into \emph{set pieces}, which consist of 24 \emph{ticks}. At the start of a set piece, each combatant rolls a \die{d12}, adding their \stat{Wit} plus 6. That's their \emph{starting tick}.

To keep track of ticks, divide a sheet of paper into sections numbered from \textbf{1} to \textbf{24}, and place tokens representing each combatant at their starting tick. Play progresses from the top of the set piece in descending order. The GM calls out each tick number, and the combatants at that number can act, in descending order of \stat{Wit}.

When it's a character's turn, their player describes what they do, to declare a \emph{sequence} of actions. For example, the description could be \textbf{``\emph{I veer my vehicle over to the rig, open my door and climb on top, jump across to the rig, and shoot my Glock at the driver}''}, which describes four actions.

Each action in the sequence takes you down the sheet by two ticks. Once you reach tick 0 or below, you can no longer act. So, if you're on tick 5, you can take a sequence of 3 actions; if you're on tick 1 or 2 you can only take one action.
