% !TEX root = ../../outback-basic-and-extended.tex

Attacks to a vehicle damage the vehicle, not its occupants, unless the attack states otherwise. Some actions can cause \emph{afflictions}, which are conditions that adversely affect your stats.

When a character reaches zero hit points, they're afflicted by \emph{unconsciousness}. Unconscious characters cannot act. At the end of the setpiece, unconscious characters roll \stat{Wit} against death. If the result is \textbf{less than 6}, they are \emph{presumed dead}. If it's \textbf{more than 12}, they're still kicking, and they start the next setpiece with hit points equal to how much more than 12 they rolled.

Once a character is presumed dead, they're out of action for the rest of combat. When combat ends, they make one final \stat{Wit} roll for their life. \textbf{Less than 6} and they're permanently dead. \textbf{6 or more} and they live to fight another day, recovering a single hit point. Any unconscious characters also recover with 1 hit point when combat ends.

When a vehicle reaches zero hit points, it's afflicted by \emph{engine fire}. If the fire isn't repaired at the end of the setpiece, the engine explodes, and any remaining occupants are presumed dead.

As the last action in a setpiece, you can restore hit points or remove an affliction. Repairing a vehicle requires a full set of tools, but no special equipment is required for working on humans. Roll \stat{Brains} and add the target's \stat{Brawn} or \stat{Ruggedness} to the total. Either add the result to the target's hit points, or remove an affliction if the result is higher than it's \stat{Difficulty} roll.
