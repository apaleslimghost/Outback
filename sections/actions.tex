% !TEX root = ../outback-basic-and-extended.tex

Here's a few examples of actions you can take on your turn. This is not an exhaustive list! First describe what you want to do, and together with the GM determine how many actions that entails and whether any of the require rolls.

\begin{abstractsection}{Attacks}
\vspace{1ex}

\begin{describe}{Unarmed hand-to-hand}
  \prop{Maximum damage}{5}
  attack with \stat{Brawn} versus \stat{Wit}.
\end{describe}

\begin{describe}{Close-range weapon}
  attack with \stat{Brawn} versus \stat{Wit}.
\end{describe}

\begin{describe}{Long-range weapon}
  attack with \stat{Brains} versus \stat{Wit}.
\end{describe}

\begin{describe}{Shunt}
  \prop{Maximum damage}{20}
  vehicular attack with \stat{Speed} vs \stat{Speed}.
\end{describe}

\begin{describe}{Ram}
  \prop{Maximum damage}{30}
  vehicular attack with \stat{Speed} vs \stat{Ruggedness}.
\end{describe}

\begin{describe}{Sideswipe}
  \prop{Maximum damage}{10}
  vehicular attack with \stat{Handling} vs \stat{Ruggedness}.
\end{describe}
\end{abstractsection}

\begin{abstractsection}{Repairs}
\vspace{1ex}

\begin{describe}{Repair}
  Choose a vehicle you have access to or character within 3 feet. Roll \die{1d8} plus  your \stat{Brains} modifier. Add the result to the HP of the target.
\end{describe}

\begin{describe}{Remove Affliction}
  Requires Blackthumb's Tools. Choose a vehicle you have access to or character within 3 feet. Roll \stat{Brains} against the difficulty of an affliciton of the target.  If you succeed, the affliction is removed.
\end{describe}
\end{abstractsection}
