% !TEX root = ../outback-basic-and-extended.tex

Here's a few examples of actions you can take on your turn. This is not an exhaustive list! First describe what you want to do, and together with the GM determine how many actions that entails and whether any of them require rolls.

\begin{describe}{Unarmed hand-to-hand}
  \prop{Maximum damage}{5}
  attack with \stat{Brawn} versus \stat{Wit}.
\end{describe}

\begin{describe}{Close-range weapon}
  attack with \stat{Brawn} versus \stat{Wit}.
\end{describe}

\begin{describe}{Long-range weapon}
  attack with \stat{Brains} versus \stat{Wit}.
\end{describe}

\begin{describe}{Shunt}
  \prop{Maximum damage}{20}
  vehicular attack with \stat{Speed} vs \stat{Speed}.
\end{describe}

\begin{describe}{Ram}
  \prop{Maximum damage}{30}
  vehicular attack with \stat{Speed} vs \stat{Ruggedness}.
\end{describe}

\begin{describe}{Sideswipe}
  \prop{Maximum damage}{10}
  vehicular attack with \stat{Handling} vs \stat{Ruggedness}.
\end{describe}

\hr

\begin{describe}{Boarding}
  If you are on the outside of a vehicle, e.g. on its roof or flatbed, you can attempt to board another vehicle in the convoy. Roll \stat{Wit} vs the target's \stat{Handling}. If you succeed, you get on top of the target vehicle. If you fail, you fall off, but you can spend an action to return to your vehicle.
\end{describe}

\begin{describe}{Hijack}
  If you are on the outside of a vehicle, you can attempt to take control of it. Roll \stat{Brawn} vs the driver's \stat{Wit}. If you succeed, you forcibly remove the driver and have control of the the vehicle.
\end{describe}

\begin{describe}{Sabotage}
  If you are on the outside of a vehicle, you can attempt to severly damage it. Roll \stat{Brains} vs the vehicle's \stat{Ruggedness}. If you succeed, you cause one affliction of your choice to the vehicle.
\end{describe}

\hr

\begin{describe}{Repair}
  As the last action in a setpiece, you can restore hit points to a vehicle or character. Roll \stat{Brains} and add the target's \stat{Brawn} or \stat{Ruggedness}. Add the result to the target's HP. If the result is below zero, you'll end up doing damage, so be careful repairing something if your \stat{Brains} is negative.
\end{describe}

\begin{describe}{Remove Affliction}
  As the last action in a setpiece, you can remove an affliction from a vehicle or character. Roll \stat{Brains} and add the target's \stat{Brawn} or \stat{Ruggedness}. If the result is higher than the affliction's \stat{Difficulty} roll, remove the affliction.
\end{describe}
