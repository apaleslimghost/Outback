% !TEX root = ../cheatsheet.tex

When undertaking a task that requires skill, you roll dice to see if you succeed. Your roll will be \emph{opposed} by the GM's or another player's, countering your skill with another character's, or an environment's difficulty.

The skills you use fall into three categories, which are the three \emph{stats} that sum up your character's ability. Some example skills are below, but you don't have to use one of them, or even declare which skill you use. Based on how you roleplay a situation, the GM may suggest a skill and stat for you to roll.

% !TEX root = ../../cheatsheet.tex

{\small\begin{tabular}{lll}
  Brawn         & Wit        & Brains         \\
  \hline
  Armour        & Agility    & Lore           \\
  Athletics     & Barter     & Navigation     \\
  Grappling     & Driving    & Perception     \\
  Hand-to-hand  & Initiative & Ranged weapons \\
  Melee weapons & Scavenging & Repair
\end{tabular}}


Each stat is between \textbf{0} and \textbf{20}, and has a modifier between \textbf{-5} and \textbf{+5}. A skill is written as ``\stat[Skill]{Stat}'', e.g. \stat[Lore]{Brains}.

\begin{abstractsection}{Dice rolls}
You and your opponent roll a \die{d12}, adding the modifier for the skill you are using. If the total is less than 12, that's your final score. If it's 12 or more, mark down a ``12'' somewhere, and roll again. Keep going until your roll is less than 12. Add all your twelves and your final roll together to get your final score.

If your score is higher than your opponent's, you succeed. If they are equal or yours is lower, you fail.
\end{abstractsection}
